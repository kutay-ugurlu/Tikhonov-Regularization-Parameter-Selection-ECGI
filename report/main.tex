\documentclass[draftcls, onecolumn, journal]{IEEEtran}
% \documentclass[journal]{IEEEtran}
% \documentclass[inproceedings]{article}
%\usepackage{fullpage}

%\renewcommand{\baselinestretch}{1.9}
\usepackage{graphicx}
%\usepackage{cite}
\usepackage[style=ieee, nohashothers=true, nosortothers=true, uniquelist=true, natbib=true, backend=biber, sorting=none]{biblatex}
\DefineBibliographyStrings{english}{andothers={}}

\addbibresource{project.bib}
\DeclareFieldFormat[article]{volume}{Vol. #1}
\DeclareFieldFormat[article]{number}{No. #1}
\DeclareFieldFormat[article]{pages}{p. #1}
\DeclareFieldFormat[inproceedings]{pages}{p. #1}

%\documentclass[journal]{IEEEtran}
\usepackage[a4paper, total={6in, 8.5in}, top=1in, bottom=1in, left=1in, right=1in]{geometry}
\usepackage{mathtools}
\usepackage{amssymb}
\usepackage{amsmath}
\usepackage{pythonhighlight}
\usepackage[utf8]{inputenc}
\usepackage{fancyhdr}
\usepackage{pythonhighlight}
\usepackage{changepage}
\usepackage{slashbox}
\usepackage{floatrow}
\usepackage{listings}
\usepackage[hidelinks]{hyperref}
\usepackage[T1]{fontenc}
\usepackage[utf8]{inputenc}
\usepackage[english]{babel}
\usepackage{csquotes}
\usepackage{booktabs}
\usepackage{multicol}
\usepackage{titlesec}
\usepackage{fontawesome5}
\usepackage{makecell}
\usepackage{footnote}

\setcounter{secnumdepth}{4}
\setcounter{tocdepth}{4}


\titleformat{\paragraph}
{\normalfont\normalsize\bfseries}{\theparagraph}{1em}{}
\titlespacing*{\paragraph}
{0pt}{3.25ex plus 1ex minus .2ex}{1.5ex plus .2ex}

\usepackage{caption}
\usepackage{subcaption}
\usepackage{color} %red, green, blue, yellow, cyan, magenta, black, white
\definecolor{mygreen}{RGB}{28,172,0} % color values Red, Green, Blue
\definecolor{mylilas}{RGB}{170,55,241}


\floatsetup[table]{capposition=top}

\sloppy
\definecolor{lightgray}{gray}{0.5}
\setlength{\parindent}{0pt}
\setlength{\headheight}{14pt}

\renewcommand{\headrulewidth}{.4mm} % header line width
\newcommand{\norm}[1]{\left\lVert#1\right\rVert}
\renewcommand\footnoterule{\kern-3pt \hrule width 3in \noindent \kern 2.6pt}


\pagestyle{fancy}
\fancyhf{}
\fancyhfoffset[L]{1cm} % left extra length
\fancyhfoffset[R]{1cm} % right extra length
\rhead{\bfseries Kutay U\u{g}urlu 2232841}
\lhead{ADPC for Tikhonov Regularization Parameter Choice}
\rfoot{}

\DeclarePairedDelimiter\ceil{\lceil}{\rceil}
\DeclarePairedDelimiter\floor{\lfloor}{\rfloor}

\author{Kutay U\u{g}urlu 2232841}

\begin{document}

    
\lstset{language=Matlab,%
    %basicstyle=\color{red},
    breaklines=true,%
    morekeywords={matlab2tikz},
    keywordstyle=\color{blue},%
    morekeywords=[2]{1}, keywordstyle=[2]{\color{black}},
    identifierstyle=\color{black},%
    stringstyle=\color{mylilas},
    commentstyle=\color{mygreen},%
    showstringspaces=false,%without this there will be a symbol in the places where there is a space
    numbers=left,%
    numberstyle={\tiny \color{black}},% size of the numbers
    numbersep=9pt, % this defines how far the numbers are from the text
    emph=[1]{for,end,break},emphstyle=[1]\color{red}, %some words to emphasise
    %emph=[2]{word1,word2}, emphstyle=[2]{style},    
}

\fancyfoot[C]{\thepage}

\title{\LARGE \LARGE EE798 Remote Image Formation Theory Project \newline \newline
Analysis and Reimplementation of \newline \textit{Improving the spatial solution of electrocardiographic
imaging: A new regularization parameter choice
technique for the Tikhonov method}}

\maketitle{\LARGE}
\pagebreak
\tableofcontents
\listoffigures
\listoftables
\pagebreak

\section{Introduction}

Electrocardiographic Imaging (ECGI) is a noninvasive method for reconstructing the epicardial potentials from the body surface potential mapping that can diagnose diseases such as tachycardia \cite{intini2005electrocardiographic} and atrial fibrillation \cite{figuera2016regularization,schuler2017ecg}. The number of measurement non-invasively taken from the torso surface, however, is less than the number of reconstructed cardiac sources that provides satisfactory spatial resolution for the diagnosis. Due to the inherent ill-posedness of this underdetermined problem, utilization of regularization is mandatory to achieve physiologically suitable solutions \cite{milanivc2014assessment}. Tikhonov Regularization is a widely used regularization technique in ECGI community and has been found to outperform the other methods depending on the formulation of the problem \cite*{milanivc2014assessment}. The regularization technique imposes a prior on the inverse problem solution and weights the candidate solutions with the data-fidelity term in the cost function with a regularization parameter $\lambda$. Chamorro-Servent, \textit{et al.}, proposes a new method called Automated Discrete Picard Condition(ADPC) in their study \cite*{chamorro2017improving} to automatically find a suitable regularization parameter $\lambda$. 
This project report investigates the idea reported in \textit{Improving the spatial solution of electrocardiographic
imaging: A new regularization parameter choice technique for the Tikhonov method}~\cite{chamorro2017improving}.  
\\
\\
The organization of the report is as follows:
\begin{itemize}
    \item The problem and the proposed solutions to it are briefly introduced in this section.
    \item The background of ECGI and theory of the regarding inverse problem are discussed in Section \nameref{sec:theory}.
    \item The methods, datasets and neural network architectures used in the implementation of the original paper along with modifications introduced in the speech emotion recognition tasks can be seen in Section \nameref{sec:implementation}.
    \item Section \nameref{sec:discussion} is left for the presentation of the results from conducted experiments along with the original results presented in the study and the discussion comparing the performances of the related method.
\end{itemize}

\section{Theory}\label{sec:theory}




\section{Results}\label{sec:results}
\section{Implementation}\label{sec:implementation}
\section{Results and Discussion}\label{sec:discussion}
\end{document}